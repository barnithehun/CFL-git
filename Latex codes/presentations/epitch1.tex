%----------------------------------------------------------------------------------------
%	PACKAGES AND THEMES
%----------------------------------------------------------------------------------------

\documentclass[notes]{beamer}

%Cheat-Sheet: https://www.cpt.univ-mrs.fr/~masson/latex/Beamer-appearance-cheat-sheet.pdf
\usetheme{Frankfurt}

%-- Layout --%
% font size
\usepackage[fontsize=11pt]{scrextend}
\usepackage{amsmath}
\usepackage[short]{optidef}
\usepackage[backend=bibtex,style=authoryear]{biblatex}
% page
\setbeamersize{text margin left=1cm, text margin right=1cm}
\setbeamercolor{button}{bg=blue, fg=white}
% line space 
\usepackage{setspace}
\setstretch{0.8}
% footline
\addtobeamertemplate{navigation symbols}{}{
    \usebeamerfont{footline}
    % \usebeamercolor[fg]{footline}
    \hspace{2em}
    \insertframenumber/\inserttotalframenumber    }
% headline
\setbeamercovered{transparent}

%-- Figure --%
\usepackage{graphicx}
% \usepackage{subfig}

%-- Table --%
\usepackage{tabularx,booktabs}
\newcolumntype{Y}{>{\centering\arraybackslash}X}

%-- Text --%
\usepackage{xcolor}
% \usepackage{tikz}

%-- Math --%
\usepackage{amssymb}
\usepackage{amsmath}

%-- Caption --%
\usepackage{caption}
\usepackage{subcaption}

%-- Effects --%
\usepackage{hyperref}

%----------------------------------------------------------------------------------------
%	TITLE PAGE
%----------------------------------------------------------------------------------------

\title[]{Debt Financing Strategies of Heterogeneous Firms} % The short title appears at the bottom of every slide, the full title is only on the title page

\author{Barnab\'as Sz\'ekely} % Your name

\date{\today } % Date, can be changed to a custom date

\begin{document}
\renewcommand{\arraystretch}{1.4}

\begin{frame}
\titlepage % Print the title page as the first slide
\end{frame}

%%----------------------------------------------------------------------------------------
%	PRESENTATION SLIDES
%----------------------------------------------------------------------------------------

%------------------------------------------------
\section{Intro}

\begin{frame}[label=slide2]
\frametitle{Asset-Based and Cash-Flow Based lending}
Motivation: 
\begin{itemize}
\item Economic determinants of creditors' payoffs 
\item Nature and severity of credit market frictions
\end{itemize} \vspace{4mm} 
Lian and Ma (2021, QJE): 
\begin{itemize}
\item Asset-based lending (ABL)  $\rightarrow$ secured against borrowers' (physical) assets, distress is resolved by liquidation
\item Cash flow-based lending (CFL)  $\rightarrow$ No specific assets as collateral, distress is resolved by reorganization
\end{itemize}
By volume, 80\% of US corporate debt is CFL! 
\begin{itemize}
\item Significant model implications
\end{itemize}


\end{frame}

%------------------------------------------------
\section{Evidence}


\begin{frame}
  \frametitle{Reliance to cash flow based finance}

  \begin{columns}
    \begin{column}{0.45\textwidth} % Adjust the width as needed
   Debt classification: 
   \begin{itemize}
   \item Following Lian and Ma
   \item US listed corporations 
   \end{itemize}
   I find: 73.4\% of debt is CFL  \vspace{1mm}  \\
  $$ CFL \ reliance = \dfrac{CFL \  debt}{Total \  debt} $$ \vspace{3mm} \\ 
  CFL-reliant firms are larger on average 
  \begin{itemize}
  \item   High fixed costs of CFL
  \end{itemize}        
    \end{column}
    \begin{column}{0.55\textwidth} % Adjust the width as needed
      \begin{figure}
        \includegraphics[width=\textwidth]{cdfs.png} % Adjust the path and options as needed
      \end{figure}
    \end{column}
  \end{columns}
\end{frame}


%------------------------------------------------
\begin{frame}[label=slide2]
\frametitle{CFL reliance across the firm sizes}
U-shape: consistent across sectors, robust in multivariate analysis
\begin{figure} [H]
\centering
\includegraphics[width=1 \textwidth]{smoothy2.png}
\end{figure}  
\end{frame}

\section{Model}
%------------------------------------------------
\begin{frame}[label=slide2]
\frametitle{Model setup}
Structural model - goals
\begin{itemize}
\item Reproduce CFL reliance across firms
\item See capital misallocation, reaction to aggregate shocks
\end{itemize} \vspace{5mm}

Model framework - Khan and Thomas (2013)
\begin{itemize}
\item Heterogeneous firms s.t. financial frictions
\item Representative households, competitive lender
\item Firms choose ABL or CFL $\rightarrow$ cost of each financing strategy depends on creditors' payoffs
\end{itemize}
\end{frame}

%------------------------------------------------
\begin{frame}[label=slide2]
\frametitle{Costs of borrowing: $q_{cfl}$ vs. $q_{abl}$}
Cost of external finance: 
\begin{itemize}
\item Inverse gross interest rate $q_i$; risk free rate: $q_0 = 1 / \beta$
\item Competitive lender $\rightarrow$ zero profit condition  
\end{itemize} \vspace{4mm}
\textbf{ABL}: the lender liquidates assets subject to a variable cost:
$$ q_{abl}(k',b',\varepsilon)b' = \beta \left[ (1-P_\chi) b' + P_\chi \min\{b', \ \phi_k (1-\delta) k' \} \right]  $$ \vspace{2mm} \\
\textbf{CFL}: the lender takes over the entire firm and resells it to the household subject to a variable \textbf{and} a fixed cost ($\zeta$)
$$ q_{cfl}(k',b',\varepsilon)b' = \beta \left[ (1-P_\chi) b' + P_\chi \min\{b', \ \phi_v V_1(k',b',\varepsilon') - \zeta \} \right]  $$ 
\end{frame}

\section{Fixed costs}
%------------------------------------------------
\begin{frame}[label=slide2]
\frametitle{The fixed cost of CFL - explaining $\zeta$ }
In-default:  reorganization cost
\begin{itemize}
\item ABL: liquidation (Chapter 7) 
\item CFL: reorganization (Chapter 11) $\rightarrow$ costly: negotiation between debtors, creditors, legal fees etc.
\end{itemize} \vspace{3mm}
No default: monitoring costs
\begin{itemize}
\item ABL: only a periodic appraisal of assets
\item CFL: lender must carry out `due diligence' on an ongoing basis
\end{itemize} 	\vspace{3mm}
Summarized by parameter $\zeta$, paid by the lender in case of default

\end{frame}


\end{document}
